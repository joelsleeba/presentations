% --- LaTeX Presentation Template - S. Venkatraman ---

% --- Set document class ---

% Remove "handout" when presenting to include pauses
\documentclass[handout, dvipsnames]{beamer}
\usetheme{default}

% Make content that is hidden by pauses "transparent"
\setbeamercovered{transparent}

% --- Slide layout settings ---

% Set line spacing
\renewcommand{\baselinestretch}{1.15}

% Set left and right text margins
\setbeamersize{text margin left=12mm, text margin right=12mm}

% Add slide numbers in bottom right corner
\setbeamertemplate{footline}[frame number]

% Remove navigation symbols
\setbeamertemplate{navigation symbols}{}

% Allow local line spacing changes
\usepackage{setspace}

% Change itemized list bullets to circles
\setbeamertemplate{itemize item}{$\bullet$}
\setbeamertemplate{itemize subitem}{$\circ$}

% --- Color and font settings ---

\usepackage{xcolor}

% Slide title background color
\definecolor{background}{HTML}{ede6d8}

% Slide title text color
\definecolor{titleText}{HTML}{B40404}

% Other possible color schemes

% - Light green/dark green -
%\definecolor{background}{HTML}{e4ede4}
%\definecolor{titleText}{HTML}{2e592f}

% - Light blue/dark blue -
%\definecolor{background}{HTML}{d5d9e8}
%\definecolor{titleText}{HTML}{2d375e}

% - Beige/dark blue -
%\definecolor{background}{HTML}{e8e2d5}
%\definecolor{titleText}{HTML}{2d3375}

% Set colors
\setbeamercolor{frametitle}{bg=background, fg=titleText}
\setbeamercolor{subtitle}{fg=titleText}

% Set font sizes for frame title and subtitle
\setbeamerfont{frametitle}{size=\fontsize{15}{16}}
\setbeamerfont{framesubtitle}{size=\small}

% --- Math/Statistics commands ---

% Add a reference number to a single line of a multi-line equation
% Usage: "\numberthis\label{labelNameHere}" in an align or gather environment
\newcommand\numberthis{\addtocounter{equation}{1}\tag{\theequation}}

% Shortcut for bold text in math mode, e.g. $\b{X}$
\let\b\mathbf

% Shortcut for bold Greek letters, e.g. $\bg{\beta}$
\let\bg\boldsymbol
% Shortcut for calligraphic script, e.g. %\mc{M}$
\let\mc\mathcal

% \mathscr{(letter here)} is sometimes used to denote vector spaces
\usepackage[mathscr]{euscript}

% Convergence: right arrow with optional text on top
% E.g. $\converge[p]$ for converges in probability
\newcommand{\converge}[1][]{\xrightarrow{#1}}

% Weak convergence: harpoon symbol with optional text on top
% E.g. $\wconverge[n\to\infty]$
\newcommand{\wconverge}[1][]{\stackrel{#1}{\rightharpoonup}}

% Equality: equals sign with optional text on top
% E.g. $X \equals[d] Y$ for equality in distribution
\newcommand{\equals}[1][]{\stackrel{\smash{#1}}{=}}

% Normal distribution: arguments are the mean and variance
% E.g. $\normal{\mu}{\sigma}$
\newcommand{\normal}[2]{\mathcal{N}\left(#1,#2\right)}

% Uniform distribution: arguments are the left and right endpoints
% E.g. $\unif{0}{1}$
\newcommand{\unif}[2]{\text{Uniform}(#1,#2)}

% Independent and identically distributed random variables
% E.g. $ X_1,...,X_n \iid \normal{0}{1}$
\newcommand{\iid}{\stackrel{\smash{\text{iid}}}{\sim}}

% Sequences (this shortcut is mostly to reduce finger strain for small hands)
% E.g. to write $\{A_n\}_{n\geq 1}$, do $\bk{A_n}{n\geq 1}$
\newcommand{\bk}[2]{\{#1\}_{#2}}

% Math mode symbols for common sets and spaces. Example usage: $\R$
\newcommand{\R}{\mathbb{R}}	% Real numbers
\newcommand{\C}{\mathbb{C}}	% Complex numbers
\newcommand{\Q}{\mathbb{Q}}	% Rational numbers
\newcommand{\Z}{\mathbb{Z}}	% Integers
\newcommand{\N}{\mathbb{N}}	% Natural numbers
\newcommand{\F}{\mathcal{F}}	% Calligraphic F for a sigma algebra
\newcommand{\El}{\mathcal{L}}	% Calligraphic L, e.g. for L^p spaces

% Math mode symbols for probability
\newcommand{\pr}{\mathbb{P}}	% Probability measure
\newcommand{\E}{\mathbb{E}}	% Expectation, e.g. $\E(X)$
\newcommand{\var}{\text{Var}}	% Variance, e.g. $\var(X)$
\newcommand{\cov}{\text{Cov}}	% Covariance, e.g. $\cov(X,Y)$
\newcommand{\corr}{\text{Corr}}	% Correlation, e.g. $\corr(X,Y)$
\newcommand{\B}{\mathcal{B}}	% Borel sigma-algebra

% Other miscellaneous symbols
\newcommand{\tth}{\text{th}}	% Non-italicized 'th', e.g. $n^\tth$
\newcommand{\Oh}{\mathcal{O}}	% Big-O notation, e.g. $\O(n)$
\newcommand{\1}{\mathds{1}}	% Indicator function, e.g. $\1_A$

% Additional commands for math mode
\DeclareMathOperator*{\argmax}{argmax}	% Argmax, e.g. $\argmax_{x\in[0,1]} f(x)$
\DeclareMathOperator*{\argmin}{argmin}	% Argmin, e.g. $\argmin_{x\in[0,1]} f(x)$
\DeclareMathOperator*{\spann}{Span}	% Span, e.g. $\spann\{X_1,...,X_n\}$
\DeclareMathOperator*{\bias}{Bias}	% Bias, e.g. $\bias(\hat\theta)$
\DeclareMathOperator*{\ran}{ran}		% Range of an operator, e.g. $\ran(T) 
\DeclareMathOperator*{\dv}{d\!}		% Non-italicized 'with respect to', e.g. $\int f(x) \dv x$
\DeclareMathOperator*{\diag}{diag}	% Diagonal of a matrix, e.g. $\diag(M)$
\DeclareMathOperator*{\trace}{trace}	% Trace of a matrix, e.g. $\trace(M)$
\DeclareMathOperator*{\supp}{supp}	% Support of a function, e.g., $\supp(f)$

% --- Presentation begins here ---

\begin{document}

% --- Title slide ---

\title{\color{titleText}Functional Calculus}
% \subtitle{\color{Blue}(No proofs, just the ideas)}
\author{Joel Sleeba\vspace{-.3cm}}
\date{\small October 22, 2023}
% \institute{IISER Thiruvananthapuram}

\begin{frame}
\titlepage
\vspace{-1.2cm}
\begin{center}
%   Things I learned during summer
% \end{spacing}}
\end{center}
\end{frame}

% --- Main content ---
\section{Basics}

% Example slide: use \pause to sequentially unveil content
\begin{frame}{Definition}
\framesubtitle{C* algebra}
\begin{enumerate}
  \item A unital algebra (vector space + multiplication) $\mathcal{A}$
    \begin{itemize}
      \item $(ab)c = a(bc)$
      \item $(a+ \lambda b)c = ac + \lambda bc$
      \item $a(b+ \lambda c) = ab + \lambda ac$
      \item $1_\mathcal{A} a = a = a1_\mathcal{A}$
    \end{itemize}
    \pause
  \item with an \textcolor{red}{involution} $*$
    \begin{itemize}
      \item $(\alpha a+\beta b)^* = \overline{\alpha}a^* + \overline{\beta}b^*$
      \item $(ab)^* = b^*a^*$
      \item $a^{**} = a$
    \end{itemize}
    \pause
  \item and a complete norm $\|\cdot \|$ that satisfy
    \begin{itemize}
      \item $\|ab\| \le \|a\| \|b\|$ (submultiplicativity)
      \item $\|a^*\| = \|a\|$
      \item $\|a^*a\| = \|a\|^2$
    \end{itemize}
\end{enumerate}
\end{frame}

% Example slide 2: Image
\begin{frame}{Examples}
\framesubtitle{C* algebra}
\begin{enumerate}
  \item $\mathbb{C}$ with standard multiplication, conjugation, and standard norm. \pause
  \item B(X), complex valued bounded functions on $X$, with pointwise multiplication, conjugation, and supremum norm.
  \item $B(\mathcal{H})$ for a Hilbert space $\mathcal{H}$ with composition, adjoint, and operator norm.
\end{enumerate}
\end{frame}

\begin{frame}{Spectrum}
\framesubtitle{C* algebra}
\begin{definition}[Invertible elements of $\mathcal{A}$]
    An element $a \in \mathcal{A}$ is called invertible if there is an element $z \in \mathcal{A}$ such that $az = 1_\mathcal{A} = za$. \pause
    We denote the collection of invertible elements of $\mathcal{A}$ by $G(\mathcal{A})$
\end{definition} \pause

\begin{definition}[Spectrum of an element]
  We define the spectum of an element $a \in \mathcal{A}$ as the collection $$\sigma(a) = \{ \lambda \in \mathbb{C} \ |  \ 1_A \lambda - a \notin G(\mathcal{A})\}$$ \pause
  and the {\bf spectral radius} of $a$ as $$r(a) = \sup \{ |\lambda| \ | \ \lambda \in \sigma(a) \}$$
\end{definition}
\end{frame}

\begin{frame}{Properties of spectrum}
\framesubtitle{C* algebra}
\begin{itemize}
  \item $G(\mathcal{A})$ is an open in $\mathcal{A}$ \pause
  \item $r(a) \le \|a\|$ \pause
  \item $\sigma(a)$ is nonempty for all $a \in A$ (Gelfand) \pause
  \item $\sigma(a)$ is closed compact in $\mathbb{C}$ \pause
  \item $r(a) = \lim \|a^n\|^\frac{1}{n}$ (Beurling) \pause
  \item If every nonzero element in $\mathcal{A}$ is invertible, then $\mathcal{A} = \mathbb{C}1_A$ (Gelfand - Mazur)
\end{itemize}
\end{frame}

\begin{frame}{Some special elements}
\framesubtitle{C* algebra}
\begin{definition}
  Let $\mathcal{A}$ be a C* algebra, an element $a \in \mathcal{A}$ is called:
  \begin{itemize}
    \item \textbf{self adjoint / hermitian} if $a^* = a$ \pause
    \item {\bf normal} if $a^*a = aa^*$ \pause
    \item {\bf unitary} if $a^*a = 1_A = aa^*$ \pause
    \item {\bf positive} if $a$ is hermitian and $\sigma(a) \subset \mathbb{R}^+$ \pause
    \item {\bf projection} if $a^2 = a$
  \end{itemize}
\end{definition}

\pause
We can easily show that if
\begin{itemize}
  \item $a$ is hermitian, then $\sigma(a) \subset \mathbb{R}$ \pause
  \item $a$ is unitary, then $\sigma(a) \subset \mathbb{T}$ 
\end{itemize}
\end{frame}

\begin{frame}{Some Properties}
\framesubtitle{C* algebra}
  We will now use $\mathcal{A}_{sa}$ and $\mathcal{A}^+$ to denote hermitian and positive elements in $\mathcal{A}$ respectively \pause 
  \begin{itemize}
    \item Every $a \in \mathcal{A}$ can be written as $a = b + ic$ where $b, c \in \mathcal{A}_{sa}$ \pause $$b = \frac{a + a^*}{2}, c = \frac{a - a^*}{2i}$$ \pause
    \item If $a, b \in \mathcal{A}^+$, then $a+b, \alpha a \in \mathcal{A}^+$ for $\alpha \ge 0$ \pause
    \item $\mathcal{A}^+ = \{a^*a \ | \ a \in \mathcal{A}\}$ \pause
    \item $a \le b$ if $b - a \in \mathcal{A}^+$ defines a partial order in $\mathcal{A}_{sa}$
  \end{itemize}
\end{frame}

\begin{frame}{Homomorphisms}
\framesubtitle{C* algebra}
\begin{definition}[Homomorphisms between C* algebras]
  An involutive multiplicative bounded linear map between C* algebras is called a homomorphism. \pause

  Let $\phi: \mathcal{A \to B}$ be a linear map between C* algebras $\mathcal{A, B}$.
  \begin{itemize}
    \item $\phi \in B(\mathcal{A, B})$ (bounded)
    \item $\phi(ab) = \phi(a) \phi(b)$ (multiplicative)
    \item $\phi(a^*) = \phi(a)^*$ (involutive)
  \end{itemize}
\end{definition}
\pause

Some properties:
\begin{itemize} %[Some properties]
  \item If $\phi: \mathcal{A} \to \mathcal{B}$, then $\phi$ is norm decreasing. ($\|\phi(a)\| \le \|a\|$) \pause
  \item Every injective $*$-homomorphisms are isometric.
\end{itemize}
\end{frame}

\begin{frame}{Gelfand Spectrum}
\framesubtitle{C* algebra}
   \begin{definition}
     If $\mathcal{A}$ is a C* algebra, we define the Gelfand specturm of $\mathcal{A}$ to be the collection of all $*$ homomorphims from $\mathcal{A} \to \mathbb{C}$ and denote it by $\Omega(\mathcal{A})$. \pause
     Since $\Omega(\mathcal{A}) \subset \mathcal{A}^*$, the dual space of $\mathcal{A}$, we can endow $\Omega(\mathcal{A})$ with the weak * topology from $\mathcal{A}^*$
   \end{definition}
   \pause

   Note that by the norm decreasing property of the $*$-homomorphisms we see that $\Omega(\mathcal{A})$ is a subset of the closed unit ball of $\mathcal{A}^*$
\end{frame}

\begin{frame}{Gelfand Transform}
\framesubtitle{Abelian C* algebra}
  Assuming $\mathcal{A}$ to be abelian gives us extra results
  \begin{enumerate}
    \item $\Omega(A)$ to be compact. \pause
    \item $\sigma(a) = \Omega(A)a = \{\tau(a) \ | \ \tau \in \Omega(A) \}$
  \end{enumerate}
\pause

\begin{definition}[Gelfand Transform]
  Given any abelian C* algebra $\mathcal{A}$, we define the Gelfand transform of $a \in \mathcal{A}$ as the map $$\hat{a}: C(\Omega(A)) \to \mathbb{C}: = \hat{a}(f) = f(a)$$
\end{definition}
\end{frame}

\begin{frame}{Gelfand Representation}
\framesubtitle{Abelian C* algebra}
If the C* algebra is abelian, we can represent the abstract C* algebra with a concrete a C* algebra of continuous functions in a compact space. \pause This is given by the Gelfand representation.
\pause

\begin{theorem}[Gelfand]
  For any abelian C* algebra $\mathcal{A}$, Gelfand representation, defined as $$\mathcal{A} \to C(\Omega(\mathcal{A})): a \to \hat{a}$$
  is an isometric $*$-isomorphism.
\end{theorem}
\end{frame}

% \begin{frame}{States and Representations}
%   \framesubtitle{towards a more general representation }
%   \begin{definition}[States of a C* algebra]
%     Given a C* algebra $\mathcal{A}$, a linear functional $\phi: \mathcal{A} \to \mathbb{C}$ is called a {\bf state} if \pause
%     \begin{enumerate}
%       \item $\phi(a) \in \mathbb{R}^+$ for all $a \in \mathcal{A}^+$ (positive) \pause
%       \item $\phi(1_A) = 1$
%     \end{enumerate}
%   \end{definition}
%   \pause
%   \begin{definition}[Representation of a C* algebra]
%     Given a C* algebra $\mathcal{A}$ and a Hilbert space $H$, a map $\pi: \mathcal{A} \to B(H)$ is called a representation if it is a $*$-homomorphism. \pause If $\pi$ is injective we call the representation {\bf faithful}.
%   \end{definition}
% \end{frame}
%
% \begin{frame}{GNS Constructions}
%   \framesubtitle{towards a more general representation }
%   Given any representation $(H, \pi)$ of a C* algebra $\mathcal{A}$ to a Hilbert space $H$, it can be verified that $$\phi: \mathcal{A} \to \mathbb{C}:= a \to \langle \pi(a) \xi, \xi \rangle$$
%   is a state for any unit vector $\xi \in H$.
%
%   \pause
%
%   The reverse is also true.
%
%   \pause
%
%   Given any state $\phi$ of a C* algebra $\mathcal{A}$, we can construct a Hilbert space $H_\phi$, a map $\pi_\phi$, and a unit vector $\xi_\phi \in H_\phi$ such that $(H_\phi, \pi_\phi)$ is a representation for $\mathcal{A}$ and $$\phi(a) = \langle \pi_\phi(a) \xi_\phi, \xi_\phi \rangle_{H_\phi}$$
% \end{frame}
%
% \begin{frame}{Universal Representation}
%   \framesubtitle{the general representation}
%   \begin{definition}[Universal Representation]
%     Let $S(\mathcal{A})$ be the collection of all states of a C* algebra $\mathcal{A}$. \pause For any $\phi \in S(\mathcal{A})$ let $(H_\phi, \pi_\phi, \xi_\phi)$ be the corresponding GNS representation, Let $$H = \bigoplus_{\phi \in S(\mathcal{A})} H_\phi, \quad\quad \pi = \bigoplus_{\phi \in S(\mathcal{A})} \pi_\phi, \quad \quad \xi = \bigoplus_{\phi \in S(\mathcal{A})} \xi_\phi$$
%     Then $(H, \pi)$ is called the universal representation of the C* algebra $\mathcal{A}$.
%   \end{definition}
% \end{frame}
%
% \begin{frame}{Gelfand Naimark Theorem}
%   \framesubtitle{the general representation}
%   \begin{theorem}[Gelfand Naimark Theorem]
%     Given any C* algebra $\mathcal{A}$, its universal representation is faithful.
%   \end{theorem}
%   \pause
%   \vspace{1em}
%   Along with the fact that injective $*$-homomorphisms are isometric and image of a C* homomorphism is a C* subalgebra, we see that every C* algebra is isometrically isomorphic to a subalgebra of operators in a Hilbert space.
% \end{frame}

\begin{frame}{Before Functional Calculus}
  \begin{lemma}[Theorem 2.1.11, Murphy]
    Let $\mathcal{B}$ be a C* subalgebra of $\mathcal{A}$. Then if $b \in \mathcal{B}$, $$\sigma_{\mathcal{A}}(a) = \sigma_{\mathcal{B}}(a)$$ 
  \end{lemma}
\end{frame}

\begin{frame}{Functional Calculus}
  \framesubtitle{Definition}
  \textbf{Motivation}: If $p \in \mathbb{C}[z, \bar{z}]$ is a polynomial and $a \in \mathcal{A}$, then $p(a)$ is a well defined element of $\mathcal{A}$. Can we generalize this to continuous functions?
  \pause
  \begin{theorem}
    Let $a$ be a normal element in a unital C* algebra $\mathcal{A}$. Let $z: \sigma(a) \to \mathcal{C}$ be the inclusion map. Then there is a unique unital $*$-homomorphism from $\phi_a: C(\sigma(a)) \to \mathcal{A}$ such that $\phi_a: z \to a$. \pause Moreover image of $\phi_a$ is the C* algebra generated by $a$ and $1_\mathcal{A}$. (Note that this C* algebra is abelian since $a^*a = aa^*)$
  \end{theorem}
\end{frame}

\begin{frame}{Functional Calculus}
  \framesubtitle{Definition}
  \begin{definition}[Functional calculus]
    Let $a$ be a normal element of a unital C* algebra $\mathcal{A}$ and $f \in C(\sigma(a))$, then we call $\phi_a$ the functional calculus of $a$.
  \end{definition}
  \pause
  \textbf{Notation}: For ease of notation when $a \in \mathcal{A}$ and $f \in C(\sigma(a))$, we will write $f(a)$ instead of $\phi_a(f)$.

  \begin{theorem}[Spectral Mapping Theorem]
    Let $a \in \mathcal{A}$ be normal and $f \in C(\sigma(a))$ then $$\sigma(f(a)) = f(\sigma(a))$$
  \end{theorem}
\end{frame}

\begin{frame}{Functional Calculus}
  \framesubtitle{Proof of Spectral Mapping}
  \begin{proof}
    Let $\mathcal{B}$ be the C* subalgebra generated by $1_\mathcal{A}, a$. Then $\mathcal{B}$ is abelian. Moreover $$\sigma(f(a)) = \{\tau(f(a)) \ | \tau \in \Omega(B)\} = \{f(\tau(a)) \ | \tau \in \Omega(B) \} = f(\sigma(a))$$
    Note that $f(\tau(a)) = \tau(f(a))$ since $f \to \tau(f(a))$ and $f \to f(\tau(a))$ are C* homomorphisms from $C(\sigma(a)) \to C^*(1, a)$ which agree for $f = 1, z$ and $C(\sigma(a))$ is generated by $1, z$. Hence it holds true for all $f \in C(\sigma(a))$
  \end{proof}
\end{frame}

\begin{frame}{Functional Calculus}
  \framesubtitle{Applications}
  \begin{lemma}
    Every self adjoint element can be written as the difference of two positive elements.
  \end{lemma}
  \begin{proof}
    Consider $f^+, f^- \in C(\sigma(a))$ where $a \in \mathcal{A}_{sa}$. where $$f^+(x) = \begin{cases}
    x,  \ x \ge 0 \\
    0,  \ x < 0
  \end{cases}
   f^-(x) = \begin{cases}
     0, \ x \ge 0 \\
     -x, \ x < 0
   \end{cases}$$
  See that $a = f^+(a) - f^-(a)$ and that this proves it.
  \end{proof}
\end{frame}

\begin{frame}{Functional Calculus}
  \framesubtitle{Applications}
  \begin{theorem}[Theorem 2.1.15, Murphy]
    Let $\Gamma$ be compact Hausdorff. For $\gamma \in \Gamma$, let $\delta_\gamma$ be a character for $C(\Gamma)$ given by $f \to f(\gamma)$. Then the map $\Gamma \to \Omega(C(\Gamma)): \gamma \to \delta_\gamma$ is a homeomorphism.
  \end{theorem}
\end{frame}


% --- Bibliography slide ---

\begin{frame}{References}
\begin{thebibliography}{10}
\beamertemplatearticlebibitems
{\small

\bibitem{Paper1}
C* Algebras and Operator Theory
\newblock Gerald Murphy \\

\bibitem{Paper 2}
Functional Analysis; Spectral Theory
\newblock V. S. Sunder\\

\bibitem{Paper 3}
  For More
\newblock \url{https://joelsleeba.github.io/resources/}
}
\end{thebibliography}
\end{frame}

% --- Thank you slide ---

\begin{frame}
\begin{center}
{\large\color{titleText} Thank you for listening!}
\vspace{1cm}

Joel Sleeba \\[1em]
joelsleeba1@gmail.com \\
  \href{https://joelsleeba.github.io}{joelsleeba.github.io}
\end{center}
\end{frame}

% --- Presentation ends here ---

\end{document}
